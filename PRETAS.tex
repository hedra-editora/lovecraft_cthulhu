\textbf{H.\,P. Lovecraft} demonstrou desde a infância interesse pelas artes e
pelas ciências, logo se tornando um leitor precoce. Aos dois anos, viu o pai ser internado em um manicômio, onde permaneceria até
morrer cinco anos mais tarde. Lovecraft continuou morando com a família,
porém a mãe jamais se recuperou da perda e começou a sofrer 
distúrbios mentais que afetaram profundamente sua relação com o filho,
criando laços doentios entre os dois. Após uma crise nervosa em 1908,
quando ainda estava em idade escolar, Lovecraft abandonou para sempre os
estudos e passou a levar uma existência reclusa até que, em 1914,
descobriu o jornalismo amador. A partir de então principiou a publicar
contos de horror e variados artigos em diversos periódicos, bem como a dedicar-se à vasta
correspondência que manteria ao longo de toda a vida.
Depois de perder a mãe em 1921 e de se casar em 1924, passou uma temporada
de penúria em Nova York; esta, somada ao fracasso de seu casamento,
obrigou-o a voltar para a casa de suas tias em Providence.
Mesmo sem nunca ter alcançado a fama em vida, foi nesta época que
Lovecraft escreveu alguns dos contos responsáveis por seu renome atual,
como este \emph{O chamado de Cthulhu}.
Morreu em 1937, vítima de câncer do intestino.

\textbf{O chamado de Cthulhu} apresenta a figura mais popular de Lovecraft, centro da série sobre os Grandes Antigos, as gigantescas e incompreensíveis criaturas anteriores a esta Terra. É a cristalização, numa imagem, de um tipo específico de terror chamado ``cósmico'': mas um cósmico íntimo e literário. Em seu Cthulhu, um monstro que dorme no fundo do mar --- verde, sombrio, doentio, descomunal e de dimensões inqualificáveis ---, o autor procedeu a uma metamorfose do próprio Kraken, monstro marinho e cefalópode da mitologia escandinava, para encontrar um código de seus próprios horrores: mas que funcionou bem, porque o verdadeiro mergulho no medo de um é o mergulho no medo de todos. Um dos grandes clássicos de horror do século \textsc{xx}, \textit{O chamado de Cthulhu} permite um desconcertante passeio pelo universo macabro de um dos grandes mestres do horror.

 
\textbf{Dirceu Villa} é poeta, tradutor e ensaísta. Publicou cinco livros de poesia, \emph{\textsc{mcmxcviii}} (1998), \emph{Descort} (2003), \emph{Icterofagia} (2008), \emph{Transformador} (2014), \emph{speechless tribes: três séries de poemas incompreensíveis}, e, entre outros, traduziu \emph{Um anarquista e outros contos}, de Joseph Conrad (2009), \emph{Lustra}, de Ezra Pound (2011), \emph{Famosa na sua cabeça}, de Mairéad Byrne (2015) e \emph{O Anjo Heurtebise}, de Jean Cocteau (2020). Doutor em Literaturas de Língua Inglesa pela \textsc{usp} (com estágio de doutorado em Londres), estudando o Renascimento na Inglaterra e na Itália, e pós-doutorado em Literatura Brasileira. Há sete anos é professor da Oficina de Tradução Poética da Casa Guilherme de Almeida (Centro de Estudos de Tradução Literária).

